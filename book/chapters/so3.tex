\chapter{\texorpdfstring{$\SOthree$}{SO(3)}: The 3D Rotation Group} As with $\SOtwo$ we denote by $\bR$ an element of $\SOthree$; these matrices are commonly known as \emph{3D rotation matrices}. Also like with $\SOtwo$ we define an action on $\mathbb{R}^3$ to be that of  matrix multiplication; for $\bu \in \mathbb{R}^3$
\begin{equation}
  \left \langle \bR, \bu \right \rangle_\SOthree = \bR \cdot \bu,
\end{equation}
which corresponds to a rotation of $\bR$.

The Lie algebra $\sothree$ consists of skew-symmetric matrices
\begin{equation}
  \sothree = \left\{ \begin{bmatrix} 0 & -\omega_z & \omega_y \\ \omega_z & 0 & -\omega_x \\ -\omega_y & \omega_x & 0 \end{bmatrix} \right\}.
\end{equation}
It is defined by just three elements $\bomega = \begin{bmatrix} \omega_x \\ \omega_y \\ \omega_z \end{bmatrix}$ which means that the hat and vee maps can be defined as
\begin{center}
  \begin{tikzpicture}
    \node (a1) {$\mathbb{R}^{3} \ni \begin{bmatrix}  \omega_x \\ \omega_y \\ \omega_z \end{bmatrix}$};
    \node at (5, 0) (a2) {$\begin{bmatrix} 0 & -\omega_{z} & \omega_y \\ \omega_{z} & 0 & -\omega_x \\ -\omega_z & \omega_x & 0 \end{bmatrix} \in \sotwo$};
    \draw[-latex] (a1) to[bend left] node[above] {$\wedge$} (a2.north west);
    \draw[-latex] (a2.south west) to[bend left] node[above] {$\vee$} (a1);
  \end{tikzpicture}
\end{center}
Matrices on this form have several interesting properties. First of all, for $\bu \in \mathbb{R}^3$ left matrix multiplication of $\hat {\bomega}$ is equivalent to taking the vector cross product: $\hat {\bomega} \bu = \bomega \times \bu$.  As a result many properties of the cross product are inherited by the embedding $\mathbb{R}^3 \overset{\wedge}\mapsto \mathbb{R}^{3 \times 3}$.

\begin{properties}[title=Properties of $\wedge$ on $\mathfrak{so}(3)$]
  For $\a, \symbf b, \symbf c \in \mathbb{R}^3$:
  \begin{subequations}
    \begin{align}
      \label{eq:so3_pow3}
      \hat \a \hat \b \hat \a            & = - (\a \cdot \b)  \hat \a,                                                                                     \\
      \label{eq:so3_hat_neg}
      \hat {\a} \symbf{b}                & = - \hat {\symbf b} \a,                                                                                                              \\
      \a \cdot (\hat{\symbf b} \symbf c) & = \symbf b \cdot (\hat{\symbf c} \a),                                                                                                \\
      A \hat {\symbf b}                  & = \tr(A) \hat{\symbf b} - (A \symbf b)^{\wedge} - \hat {\symbf{b}} A, \quad A \; \text{symmetric $3 \times 3$ matrix},               \\
      \a \cdot \symbf b                  & = \frac{1}{2} \left \langle \hat {\a}, \hat {\symbf b} \right \rangle_F = -\frac{1}{2}\tr \left( \hat {\a}, \hat {\symbf b} \right).
    \end{align}
  \end{subequations}
\end{properties}
\begin{proof}[Proof of \eqref{eq:so3_pow3}]
  Consider $\hat \a \hat \b \hat \a \c = \a \times (\b \times (\a \times \c))$. Expanding with the vector triple product gives
  \begin{equation}
    \hat \a \hat \b \hat \a \c = \a \times ((\b \cdot \c) \a - (\a \cdot \b) \c) = - (\a \cdot \b) \a \times \c = -(\a \cdot \b) \hat \a \c.
  \end{equation}
\end{proof}

\section{Formulas}

\paragraph{Adjoint}
Take any $\bu, \bv \in \mathbb{R}^3$, since the cross product satisfies $(\bR \bu) \times (\bR \bv) = \bR (\bu \times \bv)$ it follows that
\begin{equation}
  (\bR \bomega^\wedge) \bu = (\bR \bomega) \times \bu = \bR (\bomega \times \bR^T \bu) = \bR \bomega^\wedge \bR \bu,
\end{equation}
and hence $(\bR \bomega)^\wedge = \bR \bomega^\wedge \bR^T$. From \eqref{eq:def_bad} therefore
\begin{equation}
  \bAd_\bR \bomega \coloneq (\bR \bomega^\wedge \bR^T)^\vee = \bR \bomega,
\end{equation}
so $\bAd_\bR = \bR$.

\paragraph{Exponential and Logarithm}

We can use \eqref{eq:so3_pow3} to obtain the exponential map on $\SOthree$:
\begin{equation}
  \begin{aligned}
    \exp \bomega \coloneq \Exp \; \hat {\bomega} = \sum_{k \geq 0} \frac{(\hat {\bomega})^k}{k!} = I + \hat{\bomega} + \frac{\hat{\bomega}^2}{2!} - \| \bomega \|^2 \left( \frac{\hat {\bomega}}{3!} + \frac{\hat {\bomega}^2}{4!} \right) + \| \bomega \|^4 \left( \frac{\hat {\bomega}}{5!} + \frac{\hat {\bomega}^2}{6!} \right) + \ldots \\
    = I + \left( 1 - \frac{\|\bomega \|^2}{3!} + \frac{\|\bomega \|^4}{5!} - \ldots \right) \hat{\bomega} + \left( \frac{1}{2!} - \frac{\|\bomega \|^2}{4!} + \frac{\|\bomega \|^4}{6!} - \ldots \right) \hat{\bomega}^2                                                                                                                     \\
    = I + \frac{\sin \| \bomega \|}{\|\bomega \|} \hat {\bomega} + \frac{1 - \cos \| \bomega \|}{\| \bomega \|^2} \hat {\bomega}^2.
  \end{aligned}
\end{equation}
To obtain the logarithm the expression
\begin{equation}
  \bR = I_3 + \frac{\sin \| \bomega \|}{\|\bomega \|} \hat {\bomega} + \frac{1 - \cos \| \bomega \|}{\| \bomega \|^2} \hat {\bomega}^2
\end{equation}
should be inverted. First note that due to $\hat \bomega$ being skew-symmetric:
\begin{equation}
  \label{eq:so3_log_deriv1}
  \bR - \bR^T =  \frac{\sin \| \bomega \|}{\|\bomega \|} \hat {\bomega} + \frac{1 - \cos \| \bomega \|}{\| \bomega \|^2} \hat {\bomega}^2 - \frac{\sin \| \bomega \|}{\|\bomega \|} \hat {\bomega} ^T - \frac{1 - \cos \| \bomega \|}{\| \bomega \|^2} (\hat {\bomega}^T)^2 = 2 \frac{\sin \|\bomega \|}{\| \bomega \|} \hat \bomega.
\end{equation}
Secondly,
\begin{equation}
  \label{eq:so3_log_deriv2}
  \tr \left(\bR  \right) = 3 + \frac{1 - \cos \| \bomega \|}{\|\bomega \|^2} \tr(\hat \bomega^2) =3 - 2 \frac{1 - \cos \| \bomega \|}{\|\bomega \|^2} \| \bomega \|^2 = 1 + 2 \cos \| \bomega \|,
\end{equation}
which makes it possible to write down an expression for the logarithm.
\begin{equation}
  \log \bR = \left(\frac{\alpha}{\sin \alpha} \frac{\bR - \bR^T}{2}\right)^\vee, \quad \alpha = \arccos \left( \frac{\tr\bR) - 1}{2} \right).
\end{equation}

\paragraph{Derivatives of the Exponential}

We know from \eqref{eq:so3_pow3} that $[\bomega, \bomega'] = \bomega^\wedge \bomega'$, hence that $\ad_\bomega = \hat \bomega$, and from \eqref{eq:so3_pow3} that $\hat \bomega^3 = - \| \bomega \|^2 \hat \bomega$. Thus $\ad_\bomega^3 = -\| \bomega \|^2 \ad_\bomega$ and we get,
\begin{equation*}
  \begin{aligned}
    \sum_{n=0}^\infty
     &
    \frac{B_n (-1)^n}{n!} \ad_\bomega^n
    = \sum_{n=0}^\infty \frac{B_n (-1)^n}{n!} \hat \bomega^n = I_3 + \frac{\ad_\bomega}{2} + \sum_{n \geq 2} \frac{B_n (-1)^n}{n!}\ad_\bomega^n                                                                                                                                                                                                                    \\
     & = I_3 + \frac{\ad_\bomega}{2} + \left( \frac{B_2}{2!}\ad_\bomega^2 - \frac{B_4 \| \bomega \|^2} {4!} \ad_\bomega^2 + \frac{B_6 \| \bomega \|^4}{6!} \ad_\bomega^2 - \ldots \right) = I_3 + \frac{\ad_\bomega}{2} - \frac{1}{\| \bomega \|^2} \sum_{n \geq 1} \frac{B_{2n} (-1)^n \| \bomega \|^{2n}}{(2n)!} \ad_\bomega^2                                   \\
     & \overset{\eqref{eq:bernoulli_cot}}= I_3 + \frac{\ad_\bomega}{2} - \frac{1}{\| \bomega \|^2} \left( \frac{\| \bomega \|}{2} \cot \left( \frac{\| \bomega \|}{2} \right) - 1 \right) \ad_\bomega^2 = I_3 + \frac{\ad_\bomega}{2} + \left( \frac{1}{\| \bomega \|^2} - \frac{1 + \cos \| \bomega \| }{2 \|\bomega \| \sin \|\bomega \|} \right) \ad_\bomega^2,
  \end{aligned}
\end{equation*}
where the half-angle formula $\cot x = (1 + \cos x) / \sin x$ has been used. The left jacobian $\dl \exp_\bomega$ was already calculated in \eqref{eq:so3_leftjac} and since $\left( \dr \exp_{\bomega} \right)^{-1} = \left[ \left( \dl \exp_{\bomega} \right)^{-1} \right]^T$. Due to the anti-symmetry of $\ad_\bomega$ it follows that also $\mathrm{d}^r \exp_\bomega = \left[ \dl \exp_\bomega \right]^T$ must hold.

\begin{properties}[breakable, title={$\SOthree$ formula sheet}]
  \paragraph{Group definition}
  \begin{equation}
    \SOthree = \left\{ \bR \in \GL(3, \mathbb{R}) \mid \bR \bR^{T} = \bR^{T} \bR = I_{3}, \det \bR = 1 \right\}.
  \end{equation}

  \paragraph{Action on $\mathbb{R}^3$:} Left multiplication by rotation matrix
  \begin{equation}
    \left \langle \bR, \bu \right \rangle_\SOthree = \bR \bu.
  \end{equation}

  \paragraph{Algebra parameterization}
  \begin{equation}
    \left\{ \bomega = \begin{bmatrix} \omega_{x} \\ \omega_{y} \\ \omega_{z} \end{bmatrix}
    \mid \omega_{x}, \omega_{y}, \omega_{z} \in [-\pi, \pi] \right\}, \quad \hat \bomega = \begin{bmatrix} 0 & -\omega_{z} & \omega_{y} \\ \omega_{z} & 0 & -\omega_{x} \\ -\omega_{y} & \omega_{x} & 0 \end{bmatrix} \in \sothree.
  \end{equation}

  \paragraph{Adjoint}
  \begin{equation}
    \label{eq:so3_Ad}
    \bAd_{\bR} = \bR
  \end{equation}

  \paragraph{Exponential and Logarithm}
  \begin{subequations}
    \begin{align}
      \label{eq:so3_exp}
      \exp \bomega & = I + \frac{\sin \| \bomega \|}{\|\bomega \|} \hat {\bomega} + \frac{1 - \cos \| \bomega \|}{\| \bomega \|^2} \hat {\bomega}^2,        \\
      \label{eq:so3_log}
      \log \bR     & = \left(\frac{\alpha}{\sin \alpha} \frac{\bR - \bR^T}{2} \right)^{\vee}, \quad \alpha = \arccos \left( \frac{\tr(\bR) - 1}{2} \right).
    \end{align}
  \end{subequations}

  \paragraph{Bracket and Lowercase adjoint}
  \begin{subequations}
    \begin{align}
      [\bomega, \bomega'] & = \hat \bomega \bomega' \\
      \ad_\bomega         & = \hat \bomega
    \end{align}
  \end{subequations}

  \paragraph{Derivatives of the Exponential}
  \begin{subequations}
    \begin{align}
      \dr \exp_{\bomega}                     & =        I_3 - \frac{1 - \cos \| \bomega \|}{\| \bomega \|^2} \hat {\bomega} + \frac{ \| \bomega \| - \sin \| \bomega \| }{\| \bomega\|^3} \hat {\bomega}^2, \\
      \dl \exp_{\bomega}                     & = I_3 + \frac{1 - \cos \| \bomega \|}{\| \bomega \|^2} \hat {\bomega} + \frac{ \| \bomega \| - \sin \| \bomega \| }{\| \bomega\|^3} \hat {\bomega}^2,        \\
      \left( \dr \exp_{\bomega} \right)^{-1} & = I_3 + \frac{\hat \bomega}{2} + \left( \frac{1}{\| \bomega \|^2} - \frac{1 + \cos \| \bomega \| }{2 \|\bomega \| \sin \|\bomega \|} \right) \hat \bomega^2, \\
      \left( \dl \exp_{\bomega} \right)^{-1} & = I_3 - \frac{\hat \bomega}{2} + \left( \frac{1}{\| \bomega \|^2} - \frac{1 + \cos \| \bomega \| }{2 \|\bomega \| \sin \|\bomega \|} \right) \hat \bomega^2.
    \end{align}
  \end{subequations}
\end{properties}

\section{Parameterization via Isomorphism with \texorpdfstring{$\Sthree$}{S3}}

As opposed to the 2D case where $\Uone$ was a fairly straightforward isomorphic choice, the situation is more complicated in three dimensions. The usual choice is the group of unit quaternions
\begin{equation}
  \Sthree = \left \{ \bq = q_w + q_x \bi + q_y \bj + q_z \bk  : q_w^2 + q_x^2 + q_y^2 + q_z^2 = 1 \right\},
\end{equation}
which forms a double cover of $\SOthree$.  While the double cover mapping between $\Sthree$ and $\SOthree$ is fairly complicated, $\Sthree$ is isomorphic to $\SUtwo$ which is also a matrix Lie group. We therefore go via $\SUtwo$ to derive formulas for $\Sthree$.

\subsection{\texorpdfstring{$\SUtwo$}{SU(2)} is isomorphic to \texorpdfstring{$\Sthree$}{S3}}

We can associate a quaternion $\bq = q_w + q_x \symbf{i} + q_y \symbf{j} + q_z \symbf{k}$ with the unitary matrix
\begin{equation}
  \begin{bmatrix}
    q_w + i q_z & -q_x - i q_y \\
    q_x - i q_y & q_w - i q_z
  \end{bmatrix}
\end{equation}
for which it holds that $A_{\bq_1 * \bq_2} = A_{\bq_1} A_{\bq_2}$. Thus $\Sthree \cong \SUtwo$ and the matrix structure of $\SUtwo$ be leveraged to derive quaternion formulas.

Muliplying two elements in $\SUtwo$ yields:
\begin{equation}
  \begin{aligned}
    \begin{bmatrix}
      q_w + i q_z & -q_x - i q_y \\
      q_x - i q_y & q_w - i q_z
    \end{bmatrix} \begin{bmatrix}
      q_w' + i q_z' & -q_x' - i q_y' \\
      q_x' - i q_y' & q_w' - i q_z'
    \end{bmatrix}
    = \begin{bmatrix}
      q_w'' + i q_z'' & -q_x'' - i q_y'' \\
      q_x'' - i q_y'' & q_w'' - i q_z''
    \end{bmatrix}
  \end{aligned}
\end{equation}
where
\begin{equation}
  \label{eq:quaternion_composition}
  \begin{aligned}
    q_w'' & = q_w q_w' - q_x q_x' - q_y q_y' - q_z q_z', \\
    q_x'' & = q_x q_w' + q_w q_x' + q_y q_z' - q_z q_y', \\
    q_y'' & = q_y q_w' + q_w q_y' + q_z q_x' - q_x q_z', \\
    q_z'' & = q_z q_w' + q_w q_z' + q_x q_y' - q_y q_x',
  \end{aligned}
\end{equation}
which is exactly what is obtained by carrying out the usual quaternion multiplication
\begin{equation}
  (q_w + q_x \symbf{i} + q_y \symbf{j} + q_z \symbf{k}) * (q_w' + q_x' \symbf{i} + q_y' \symbf{j} + q_z' \symbf{k})
\end{equation}
with the quaternion rules $\symbf{i}\symbf{j} = \symbf{k}, \symbf{j}\symbf{k} = \symbf{i}, \symbf{k}\symbf{i} = \symbf{j}$ and $\symbf{i}^2 = \symbf{j}^2 = \symbf{k}^2 = -1$.

\subsection{\texorpdfstring{$\Sthree$}{S3} forms a double cover of \texorpdfstring{$\SOthree$}{SO(3)}}

A quaternion $\bq = q_w + q_x \symbf{i} + q_y \symbf{j} + q_z \symbf{k}$ acts on $\bu \coloneq \begin{bmatrix} u_x \\ u_y \\ u_z \end{bmatrix} \in \mathbb{R}^3$ as quaternion rotation $\bq * \bu * \bar {\bq}$ where $\bu$ is associated with the quaternion $u_x \symbf{i} + u_y \symbf{j} + u_z \symbf{k}$.

In terms of matrix multiplication operation can be written
\begin{equation}
  \begin{aligned}
    \begin{bmatrix}
      q_w + i q_z & -q_x - i q_y \\
      q_x - i q_y & q_w - i q_z
    \end{bmatrix} \begin{bmatrix}
      i u_z       & -u_x - i u_y \\
      u_x - i u_y & i u_z
    \end{bmatrix} \begin{bmatrix}
      q_w - i q_z   & q_x + i q_y \\
      - q_x + i q_y & q_w + i q_z
    \end{bmatrix} \\
    = \begin{bmatrix}
      i u_z'        & -u_x' - i u_y' \\
      u_x' - i u_y' & i u_z'
    \end{bmatrix}
  \end{aligned}
\end{equation}
for
\begin{equation}
  \label{eq:quat_rotation}
  \begin{aligned}
    u_x' & = (1 - 2(q_y^2 + q_z^2)) u_x + 2(q_x q_y - q_w q_z) u_y + 2( q_x q_z + q_w q_y ) u_z   \\
    u_y' & = 2(q_x q_y + q_w q_z) u_x + (1 - 2 (q_x^2 + q_z^2)) u_y + 2( q_y q_z-q_w q_x) u_z     \\
    u_z' & = 2 (q_x q_z  - q_w q_y) u_x  + 2 (q_w q_x + q_y q_z) u_y + (1 - 2(q_x^2 + q_y^2)) u_z
  \end{aligned}
\end{equation}
Since this is a linear mapping in $u_x, u_y, u_z$ we can identify $\bq$ with a matrix $\bR(\bq)$ with coefficients
\begin{equation}
  \label{eq:quat_to_so3}
  \bR(\bq) = \begin{bmatrix}
    (1 - 2(q_y^2 + q_z^2)) & 2(q_x q_y - q_w q_z)    & 2( q_x q_z + q_w q_y ) \\
    2(q_x q_y + q_w q_z)   & (1 - 2 (q_x^2 + q_z^2)) & 2( q_y q_z-q_w q_x)    \\
    2 (q_x q_z  - q_w q_y) & 2 (q_w q_x + q_y q_z)   & (1 - 2(q_x^2 + q_y^2))
  \end{bmatrix},
\end{equation}
which is a mapping from $\Sthree$ to $\SOthree$. However, the mapping is not one-to-one since both $\bq$ and $-\bq$ map to the same rotation matrix: $\bR(\bq) = \bR(-\bq)$. This is what is meant when saying that $\Sthree$ forms a \emph{double cover} of $\SOthree$.

\todo[inline]{Inverse mapping...}

\section{Formulas}

To obtain the exponential map on $\Sthree$ we go via $\SUtwo$. The Lie algebra $\mathfrak{su}(2)$ is parameterized by three elements $\bomega = (\omega_x, \omega_y, \omega_z)$ that can be associated with the skew-Hermitian matrix
\begin{equation}
  \bomega^\wedge \coloneq \frac{1}{{2}} \begin{bmatrix}
    i \omega_z            & -\omega_x - i \omega_y \\
    \omega_x - i \omega_y & -i \omega_z
  \end{bmatrix},
\end{equation}
where the factor $1/2$ is added for reasons that will become clear below. A simple calculation reveals that $\hat \bomega^2 =  -\frac{\| \bomega \|}{4} I_2$ which can be used to evaluate the exponential.
\begin{equation*}
  \begin{aligned}
    \Exp \hat \bomega
     & = \sum_{k \geq 0} \frac{\hat \bomega^k}{k!} = \sum_{k \geq 0} \frac{1}{k!} \left(-\frac{\| \bomega \|}{2} \right)^{2 * \lfloor \frac{k}{2} \rfloor} \hat \bomega^{(k \mod 2)}                      \\
     & = \left( 1 - \frac{(\|\bomega\|/2)^2}{2!} + \frac{(\|\bomega\|/2)^4}{4!} - \ldots \right) I + \left( 1 - \frac{(\|\bomega\|/2)^2}{3!} + \frac{(\|\bomega\|/2)^4}{5!} + \ldots \right) \hat \bomega \\
     & = \cos (\|\bomega\| / 2) I + 2 \frac{\sin \|\bomega\| / 2}{\|\bomega\|} \hat \bomega = \begin{bmatrix}
      \cos(\|\bomega\| / 2) + i \frac{\omega_z \sin (\|\bomega\|/2) }{\|\bomega\|} & (-\omega_x - i \omega_y) \frac{\sin \|\bomega\|/2}{\|\bomega\|}              \\
      (\omega_x - i \omega_y) \frac{\sin \|\bomega\|/2}{\|\bomega\|}               & \cos(\|\bomega\| / 2) - i \frac{\omega_z \sin (\|\bomega\|/2) }{\|\bomega\|}
    \end{bmatrix}.
  \end{aligned}
\end{equation*}
Since the mappings $\wedge: \Sthree \rightarrow \SUtwo$ and $\vee: \SUtwo \rightarrow \Sthree$ are straightforward, we can identify the exponentail and logarithm on $\Sthree$.
\begin{subequations}
  \label{eq:su2_exp_log}
  \begin{align}
    \label{eq:su2_exp}\exp (\omega_x, \omega_y, \omega_z) & = \left(
    \cos (\| \bomega \|/2), \frac{\omega_x}{\| \bomega \|} \sin (\| \bomega \|/2), \frac{\omega_y}{\| \bomega \|} \sin (\| \bomega \|/2), \frac{\omega_z}{\| \bomega \|} \sin ( \| \bomega \|/2 ) \right), \\
    \label{eq:su2_log} \log (q_w, q_x, q_y, q_z)          & = \left( 2 \frac{ \arctantwo \left(\sqrt{q_x^2 + q_y^2 + q_z^2}, q_w \right) }{ \sqrt{q_x^2 + q_y^2 + q_z^2} } \right) \times (q_x, q_y, q_z).
  \end{align}
\end{subequations}
From \eqref{eq:su2_exp_log} the reason to divide the expression for $\hat \bomega$ by a factor 2  becomes apparent---$\| \bomega \|$ represents the rotation angle in radians. We provide a quick proof for the logarithm expression.
\begin{proof}[Proof of \eqref{eq:su2_log}]
  Let $(q_w, q_x, q_y, q_z) = \exp(\omega_x, \omega_y, \omega_z)$. From \eqref{eq:su2_exp} we have that
  \begin{equation}
    \sqrt{q_x^2 + q_y^2 + q_z^2} = \sqrt{ \left( \frac{\omega_x}{\| \bomega \| } \sin ( \| \bomega \| / 2) \right)^2 + \left( \frac{\omega_x}{\| \bomega \| } \sin ( \| \bomega \| / 2) \right)^2 + \left( \frac{\omega_x}{\| \bomega \| } \sin ( \| \bomega \| / 2) \right)^2 } = \sin (\| \bomega \| / 2).
  \end{equation}
  It also follows from the same equation that
  \begin{equation}
    \begin{aligned}
      \bomega & = \frac{\|\bomega\|}{\sin(\| \bomega \| / 2)}(q_x, q_y, q_z)
      = \frac{\arctantwo(\sin \| \bomega \| , \cos \| \bomega \| )}{\sqrt{q_x^2 + q_y^2 + q_z^2}}                       \\
              & = \frac{2 \arctantwo(\sin (\| \bomega \| / 2), \cos (\| \bomega \| / 2))}{\sqrt{q_x^2 + q_y^2 + q_z^2}}
      = \frac{2 \arctantwo\left( \sqrt{q_x^2 + q_y^2 + q_z^2}, q_w \right)}{\sqrt{q_x^2 + q_y^2 + q_z^2}}.
    \end{aligned}
  \end{equation}
\end{proof}

\begin{properties}[breakable, title={$\Sthree$ formula sheet}]
  \paragraph{Group definition}
  \begin{equation}
    \Sthree = \left\{ \bq = q_w + q_x \bi + q_y \bj + q_z \bk : q_{w}^{2} + q_{x}^{2} + q_y^2 + q_{z}^{2} = 1 \right\}
  \end{equation}
  \begin{itemize}
    \item Identity element: $1$
    \item Inverse: $(q_w + q_x \bi + q_z \bj + q_w \bk)^{-1} = q_w - q_x \bi - q_z \bj - q_w \bk$
    \item Composition via \eqref{eq:quaternion_composition}
  \end{itemize}
  $\Sthree$ forms a double cover of $\SOthree$ via \eqref{eq:quat_to_so3} and therefore inherits Lie algebra properties from $\SOthree$.

  \paragraph{Rotation action on $\mathbb{R}^3$}
  Defined by \eqref{eq:quat_rotation}.

  \paragraph{Exponential and Logarithm}
  \begin{subequations}
    \begin{align}
      \label{eq:s3_exp}
      \exp \bomega & = \cos \left( \frac{\| \bomega \|}{2} \right) + \frac{1}{\| \bomega \|} \sin \left( \frac{\| \bomega \|}{2} \right) \left( \omega_x \bi + \omega_y \bj + \omega_z \bk \right), \\
      \label{eq:s3_log}
      \log \bq     & = \left( 2 \frac{ \arctantwo \left(\sqrt{q_x^2 + q_y^2 + q_z^2}, q_w \right) }{ \sqrt{q_x^2 + q_y^2 + q_z^2} } \right) \times \begin{bmatrix} q_{x} \\ q_{y} \\ q_{z} \end{bmatrix}.
    \end{align}
  \end{subequations}
\end{properties}

We conclude with some additional quaternion formulas that may be of practical interest.

\begin{properties}[title=Useful quaternion identities]
  \paragraph{Axis-angle to quaternion} The quaternion $\bq$ representing the rotation about a unit axis $\symbf{\beta} = (\beta_x, \beta_y, \beta_z)$ for an angle $\alpha$ is
  \begin{equation}
    \label{eq:axis_angle_to_quaternion}
    \bq = \cos \left( \frac{\alpha}{2} \right) + \sin \left( \frac{\alpha}{2} \right) \left( \beta_x \symbf{i} + \beta_y \symbf{j} + \beta_z \symbf{k} \right),
  \end{equation}
  which follows from \eqref{eq:s3_exp} and the interpretation of $\bomega$ as angles.

  \paragraph{Two vectors to quaternion} A quaternion $\bq$ such that $\bq \bu = \bv$ for unit vectors $\bu, \bv$.
  \begin{equation}
    \label{eq:two_vectors_to_quaternion}
    \bq = \sqrt{\frac{1 + s}{2}} + \sqrt{\frac{1 - s}{2}} \left( \beta_x \symbf{i} + \beta_y \symbf{j} + \beta_z \symbf{k} \right), \quad s = \bu \cdot \bv, \; \symbf{\beta} = \bu \times \bv.
  \end{equation}

  \paragraph{Hopf fibration} The quaternions can be parameterized as the product of a rotation $\bq_\theta$ around the $z$ axis and a quaternion that rotates $\symbf{e}_z$ to $\symbf{\beta} \coloneq \begin{bmatrix} \beta_x, \beta_y, \beta_z \end{bmatrix} \in \mathbb{S}^2$ as
  \begin{equation}
    \bq = \bq_{\symbf{\beta}} * \bq_{\theta}, \quad \bq_{\symbf{\beta}} = \frac{1}{\sqrt{2(1 + \beta_z)}} \left( 1 + \beta_z - \symbf{i} \beta_x + \symbf{j} \beta_y \right), \; \; \bq_{\theta} = \cos \left( \frac{\theta}{2} \right) + \symbf{k} \sin \left( \frac{\theta}{2} \right).
  \end{equation}
  The special case when $\beta_z = -1$ is a singularity and must be handled separately, for example by setting $\bq_{\begin{bmatrix}0, 0, -1\end{bmatrix}} = \symbf{i}$. The Hopf parameterization is a manifestation of the fact that $\Sthree$ locally is a product of the spaces $\mathbb{S}^2$ and $\mathbb{S}^1$.
\end{properties}
\begin{proof}[Proof of \eqref{eq:two_vectors_to_quaternion}]
  From properties of the dot and cross products the sought-after rotation is about the axis $\symbf{\beta} = \bu \times \bv$ for the angle $\alpha$ such that $s \coloneq \bu \cdot \bv = \cos (\alpha)$. The half-angle formulas then give that $\cos (\alpha / 2) = \sqrt{(1 + s) / 2}$, and similarly for the sine part in \eqref{eq:axis_angle_to_quaternion}.
\end{proof}

